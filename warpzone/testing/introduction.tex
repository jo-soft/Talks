\begin{frame}
  \titlepage
\end{frame}

\begin{frame}
  \frametitle{Agenda}

  \begin{itemize}
    \item Einführung
    \item Verscheiden Teststrategien
    \item Wie? Ich soll alles testen?
    \item Mocken, Fixtures, ... was ist das eigentlich?
    \item Beispiele und Probleme
  \end{itemize}
\end{frame}

\section{Einführung}
\begin{frame}
  \frametitle{Einführung}
  \framesubtitle{Übersicht und Entwicklung}
  Aus \textit{The Growth of Software Testing - D. Gelperin, B. Hetze} / \textit{History of Ideas in Software Testing} (\url{www.softwaretestpro.com}):

    \begin{itemize}
      \item Bis 1956: Debug orientiert. Also keine Unterscheidung zwischen Debuging und Testing.
        \pause
      \item 60er/70er: Trennung von Testing und Debugging. Theoretische Korrektheitsbeweise:\\
        Equivalence Classes, Boundaries, Error Guessing, Cause/Effect Graphing (It's history, good and bad) - Functional to Unit Testing (1970)
        \pause
      \item 80er: Testen mit dem Ziel Fehler zu finden.\\
        Rethinking Systems Analysis and Design -  Jerry Weinberg  (1988)
        \pause
      \item frühen 90er: Verbreitung von Bugtracking Systemen und Software Versions Kontrol Systemem steigt.
        \pause
      \item seit den späten 90er:  Aufkommen von Test Driven Development und Agiler Entwicklung
        \pause
      \item seit späten 00er: Oberflächen tests, Komplexe Hilstechnologien, Testing Frameworks usw.
        \pause
    \end{itemize}

\end{frame}

\begin{frame}
  \frametitle{Einführung}
  \framesubtitle{Test Philosophien}
  \begin{itemize}
    \item Der (End-)User testet.
      \pause
    \item Nur Tests vor Releases.
      \pause
    \item Tests sind grundlegender Bestandteil des Entwicklungsprozesses.\\
      (Test Driven Development, Behavior Driven Development)
  \end{itemize}
\end{frame}

\begin{frame}
  \frametitle{Einführung}
  \framesubtitle{Was kann man von Tests erwarten?}
  \begin{center}
    \includegraphics[scale=0.4]{unittestingCat.jpg}
  \end{center}
\end{frame}

\section{Teststrategien}
\begin{frame}
  \frametitle{Teststrategien}
  \framesubtitle{Überblick}
  \begin{block}{Welche Strategien gibt es zum Testen?}
    \pause
    \begin{itemize}
      \item Unit Tests
        \pause
      \item Integrations Tests
        \pause
      \item Oberflächen Tests
        \pause
      \item Infrastruktur Tests.
        \pause
      \item System Tests
        \pause
    \end{itemize}
  \end{block}
  \pause
  \begin{block}{Test Ansätze:}
    \begin{itemize}
      \item Blackbox Testen: Testen mit Kenntnis der Implementierung
        \pause
      \item Whitebox Testen: Testen ohne Kenntnis der Implementierung
    \end{itemize}
  \end{block}`
\end{frame}

\begin{frame}
  \frametitle{Teststrategien}
  \framesubtitle{Infrastruktur Tests}
  Testen von Verfügbarkeiten, Failover, usw.
  \begin{block}{Testbare Komponenten}
    \pause
    \begin{itemize}
      \item Rechenleistung
        \pause
      \item Speicherinfrastrutkur
        \pause
      \item Netzwerkinfrastruktur
        \pause
      \item Hardware
    \end{itemize}
  \end{block}
  \pause
  Keine Ahnung wie dies automatisiert werden kann.\\

  Weder Black- noch Whitebox Testen.
\end{frame}

\begin{frame}
  \frametitle{Teststrategien}
  \framesubtitle{Oberflächen Tests I}
  Whitebox Test von Programen über die Oberfläche.\\
  \begin{block}{Grosser Vorteil:}
    \pause
    Man benötigt keine Entwickler :)
  \end{block}
  \pause
  \begin{block}{Was ist testbar?}
    \pause
    \begin{itemize}
    \item  Existens von GUI Elementen
      \pause
    \item Korrektes Verhalten der GUI Elemente
      \pause
    \item Korrekter Ablauf des Workflows (aus Anwendersicht)
      \pause
    \item Korrektes Verhalten in verschiedenen Browsern
    \end{itemize}
  \end{block}
\end{frame}

\begin{frame}
  \frametitle{Teststrategien}
  \framesubtitle{Oberflächen Tests II}
  \begin{block}{Was ist nicht testbar}
    \pause
    \begin{itemize}
    \item Nicht direkt für den Benutzer sichtbares Verhalten (z.B. Mail Versand)
      \pause
    \item Anforderungen, welche keine Interaktion erfordern.
      \pause
    \item Sicher noch mehr.
    \end{itemize}
  \end{block}
  \pause
  \begin{block}{Frameworks}
    \begin{itemize}
      \item Selenium(Webseitentests, Bindings u.a. für Java, Python, C\#, PHP, Ruby)
      \item Windmill
    \end{itemize}
  \end{block}
\end{frame}

\begin{frame}
  \frametitle{Teststrategien}
  \framesubtitle{System Tests}
  Whitebox Test gegen die gesamten Anforderungen.
  \pause
  \begin{block}{Was genau ist das?}
    \begin{itemize}
    \item  Vereinen Oberflächen-, Integrations, sowie Infrastruktur Tests
      \pause
    \item Darum kein einheintliches Framework.
      \pause
    \item Testumgebung sollte der Produktivumgebung sehr ähnlich sein.
      \pause
    \item Oftmals zusätzlich zu automatisierten Tests.
    \end{itemize}
  \end{block}
\end{frame}


\begin{frame}
  \frametitle{Teststrategien}
  \framesubtitle{Integrations Tests I}
  Whitebox Test gegen die API Anforderungen oder von komplexer Funktionalität.
  \pause
  \begin{block}{Was ist testbar?}
    \pause
    \begin{itemize}
    \item (öffentliche) APIs
      \pause
    \item Business Logik
      \pause
    \end{itemize}
  \end{block}

  \pause
  \begin{block}{Was ist nicht testbar?}
    \pause
    \begin{itemize}
      \item Oberflächen ($\rightarrow$ Oberflächentests)
        \pause
      \item Infrastruktur ($\rightarrow$ Infrastrukturtests)
        \pause
      \item Konkrete Implementierungsdetails ($\rightarrow$ Unittests)
    \end{itemize}
  \end{block}
\end{frame}

\begin{frame}
  \frametitle{Teststrategien}
  \framesubtitle{Integrations Tests II}

  \begin{block}{Also, was macht ein Integrations Test?}
    \pause
    \begin{itemize}
      \item Testet, dass implementiert wurde, was vereinbart wurde.
        \pause
      \item Testet konkretes Verhalten
        \pause
      \item Dokumentiert Business Logik
        \pause
      \item Große Hilfe beim Refakturieren
    \end{itemize}
  \end{block}

  \pause

  \begin{block}{Frameworks}
    \pause
    \begin{itemize}
      \item Abhangig vom Einsatzgebiet und Programmiersprache
        \pause
      \item Erlauben automatisierung
        \pause
      \item Beispiele\footnote{\url{en.wikipedia.org/wiki/List\_of\_unit\_testing\_frameworks}}:
        PyTest, EUnit, JUnit, Opmock(C/C++), Jasemine (JavaScript)
    \end{itemize}
  \end{block}

\end{frame}
